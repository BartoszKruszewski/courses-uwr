% Nie jestem zbyt dobrze zaznajomiony z LaTeX'em,
% a chciałem, żeby moja praca wyglądała dobrze,
% więc skorzystałem z pomocy ChatGPT przy formatowaniu
% ostatecznego wyglądu pdf. Stąd pochodzą różne 
% podejrzane instrukcje LaTeX'a.

% Całość obliczeń niezbędnych do rozwiązania zadań
% została oczywiście wykonana ręcznie lub
% z użyciem narzędzi zbliżonych do kalkulatorów statystycznych.

\documentclass{article}
\usepackage{amsmath}
\usepackage{booktabs} 
\usepackage{siunitx} 
\sisetup{output-exponent-marker=\ensuremath{\mathrm{e}}}
\usepackage{titlesec}
\usepackage{float}
\usepackage[margin=1in,top=0.75in]{geometry}
\usepackage{caption}

\titleformat{\section}{\normalfont\Large\bfseries}{Zadanie \thesection}{1em}{}

\title{Zadanie egzaminacyjne 3}
\author{Bartosz Kruszewski - 337568}

\pagenumbering{gobble}

\renewcommand{\tablename}{}
\renewcommand{\thetable}{}
\captionsetup[table]{labelsep=none}

\begin{document}

\maketitle

\section{}
\begin{equation*}
M_x(t) = \int_{0}^{\infty} e^{tx} \lambda e^{-\lambda x} dx = \frac{\lambda}{\lambda - t}
\end{equation*}

\section{}
\begin{alignat*}{2}
&\text{Markov:} &\qquad P(X \ge \lambda a) &\le \frac{1}{\lambda^{2} a} \\
&\text{Chebyshev:} &\qquad P(X \ge \lambda a) &\le \frac{1}{(\lambda^{2} a - 1)^2} \\
&\text{Chernoff:} &\qquad P(X \ge \lambda a) &\le \lambda^{2} a e^{-\lambda^{2} a + 1}
\end{alignat*}

\section{}

\begin{equation*}
    \lambda = k + m + 1 = 8 + 6 + 1 = 15
\end{equation*}

\begin{table*}[htbp]
    \centering
    \caption{Wartości w postaci wykładniczej}
    \begin{tabular}{cccccc}
        \toprule
        Wartości $a$ & Wartości dokładne & Markov & Chebyshev & Chernoff \\
        \midrule
        3 & $e^{-15^{2} * 3}$ & $\frac{1}{15^2 * 3}$ & $\frac{1}{(15^2 * 3 - 1)^2}$ & $15^{2} * 3 e^{-15^{2} * 3 + 1}$ \\
        4 & $e^{-15^{2} * 4}$ & $\frac{1}{15^2 * 4}$ & $\frac{1}{(15^2 * 4 - 1)^2}$ & $15^{2} * 4 e^{-15^{2} * 4 + 1}$ \\
        6 & $e^{-15^{2} * 6}$ & $\frac{1}{15^2 * 6}$ & $\frac{1}{(15^2 * 6 - 1)^2}$ & $15^{2} * 6 e^{-15^{2} * 6 + 1}$ \\
        10 & $e^{-15^{2} * 10}$ & $\frac{1}{15^2 * 10}$ & $\frac{1}{(15^2 * 10 - 1)^2}$ & $15^{2} * 10 e^{-15^{2} * 10 + 1}$ \\
        \bottomrule
    \end{tabular}
\end{table*}

\begin{table*}[htbp]
    \centering
    \caption{Wartości w postaci liczbowej}
    \begin{tabular}{cccccc}
        \toprule
        Wartości $a$ & Wartości dokładne & Markov & Chebyshev & Chernoff \\
        \midrule
        3 & \num{7.0994e-294} & \num{1.4814e-3} & \num{2.2013e-6} & \num{1.3026e-290} \\
        4 & \num{1.3644e-391} & \num{1.1111e-3} & \num{1.2373e-6} & \num{3.3381e-388} \\
        6 & \num{5.4021e-587} & \num{7.4074e-4} & \num{5.4951e-7} & \num{1.8495e-583} \\
        10 & \num{6.8772e-978} & \num{4.4444e-4} & \num{1.9770e-7} & \num{4.2062e-974} \\
        \bottomrule
    \end{tabular}
\end{table*}

\end{document}
